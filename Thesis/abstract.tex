% Thesis Abstract ------------------------------------------------------

\prefacesection{Abstract}

\noindent
In this work we present a novel approach to the analysis of current grid-based path planning algorithms. Traditional path planners limit an agent's possible headings to increments of $\dfrac{\pi}{4}$ which results in unnatural and suboptimal paths that are difficult to traverse in practice. Our work compares this traditional solution to a number of alternative planners that use varying methods for heading calculation. We prove that planners which are not limited to this subset of headings are significantly more efficient even when accounting for computational times. Based on these findings we believe a significant amount of time and physical energy can be saved simply by adopting free-form natural planners in robotic agent's.

\smallskip

% ----------------------------------------------------------------------
