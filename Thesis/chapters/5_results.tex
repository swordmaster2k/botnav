\chapter{Analysis}

\noindent
The purpose of this section is to define the testing methodology used to compare D* Lite and Field D* to each other. It is split into two sections one covering the tests carried for simulations and the other for actual physical testing and their results.

%----------------------------------------------------------------------------------------------------------------------

\section{Test Cases}

\noindent
Provide all of the necessary details surrounding the sample environments used, file names, physical and cell size, and an example of one possibly a picture or a scaled text version. These environments should be based on real ones however they will remain unchanged during the planning process.

\newpage

\subsection{Test Case 1: Path Length}

\noindent 
Each of the planners produce one key output and that is a path from our start position to the goal. A property that all of these paths share in common is their physical length in metres. The goal of all of these planners is to compute the shortest path from the robot's current position to the goal. One fundamental test that we can undertake to evaluate the performance of each planner is to take the shortest suggested path and compare every other path with it. \\

\noindent
What we will be able gather from this kind of test is a percentage that tells how costly a suboptimal path is to traverse in comparison with the planner that produced the shortest path. Based on a large number of evaluations from every possible starting position to the goal for a given map it will be possible to say which planning algorithm on average produces the shortest path. This is a good indicator for when comes to deciding upon best planner however it does not account for the length of time it took to compute that optimal path. \\

\noindent
\textbf{Expected:} based on Field D*'s authors it is expected that Field D*'s result will be shorted for any paths involve headings other than $\dfrac{\pi}{4}$.\\

\noindent
\textbf{Output:} a plot file containing the full path for use with gnuplot.

\subsubsection{Evaluation}

\newpage

\subsection{Test 2: Vertex Accesses}

\noindent
\textbf{Test 2:} is concerned with how many vertices (cells) need to be accessed to build a shortest path. This is different from expansion which looks at all the vertices at once.\\

\noindent
\textbf{Expected:} since Field D* can deal with more headings it should produce a shorter path and therefore pass through less vertices than D* Lite.\\

\noindent
\textbf{Output:} absolute number of vertex accesses wrote to the debug file.

\subsubsection{Evaluation}

\newpage

\subsection{Test 3: Execution Time}

\noindent
\textbf{Test 2:} is concerned with how the time it took to execute each planning step, this value will be the total time taken for all planning steps.\\

\noindent
\textbf{Expected:} Field D*'s authors claim that on average it takes 1.7 times as long to plan than D* Lite something close to this figure is expected but it is subject to the hardware.\\

\noindent
\textbf{Output:} absolute total planning time wrote to the debug file.

\subsubsection{Evaluation}

\subsection{Test 4: Heading Changes}

\subsubsection{Evaluation}

\section{Discussion}
Discuss the findings, highlight any particularly interesting trends or patterns, also mention any difficulties encountered during testing. State the significance of the results