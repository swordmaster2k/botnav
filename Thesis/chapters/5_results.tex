\chapter{Analysis}

\noindent
The purpose of this section is to define the testing methodology used to compare D* Lite and Field D* to each other. It is split into two sections one covering the tests carried for simulations and the other for actual physical testing and their results.

%----------------------------------------------------------------------------------------------------------------------

\section{Test Cases}

\noindent
Provide all of the necessary details surrounding the sample environments used, file names, physical and cell size, and an example of one possibly a picture or a scaled text version. These environments should be based on real ones however they will remain unchanged during the planning process.

\newpage

\subsection{Test Case 1: Path Length}

\noindent 
Each of the planners produce one key output and that is a path from our start position to the goal. A property that all of these paths share in common is their physical length in metres. The goal of all of these planners is to compute the shortest path from the robot's current position to the goal. One fundamental test that we can undertake to evaluate the performance of each planner is to take the shortest suggested path and compare every other path to it. \\

\noindent
What we will be able gather from this kind of test is a percentage that tells how costly a suboptimal path is to traverse in comparison with the planner that produced the shortest path. Based on a large number of evaluations from every possible starting position to the goal for a given map it will be possible to say which planning algorithm on average produces the shortest path. This is a good indicator for when it comes to deciding upon the best planner to use however it does not account for the length of time it took to compute that optimal path. \\

\noindent
\textbf{Expected:} based on the claims of the authors of Field D* and Theta* we expect that the planners capable of dealing with headings other than $\dfrac{\pi}{4}$ will perform better.\\

\noindent
\textbf{Input:} map grid, start position, end position
\textbf{Output:} path of points OR an empty path

\subsubsection{Evaluation}

\newpage

\subsection{Test 2: Vertex Accesses}

\noindent
While execution time seems like an obvious measure that can help determine the most capable path planner it is highly subjective and variable. The time the planner spends in a processor can vary hugely based on the hardware configuration and the operating systems thread manager. For example a robot running around on a single core 700mhz processor will not have the same computing power as a quad core clocked at 2.2Ghz. The time spent calculating the path on the 700mhz processor could be significantly higher, yet the steps involved are exactly the same. \\

\noindent
An alternative to clocking the execution time is to keep track of the number of operations it takes to reach a certain goal. When it comes to path planning algorithms the highest costing operation is anything that involves accessing a \textit{vertex}. Depending of the representation a vertex can either be an entire cell or an edge, vertices are accessed during cost calculation, updates, and traversals.  \\

\noindent
\textbf{Test 2:} is concerned with how many vertices (cells) need to be accessed to build a shortest path. This is different from expansion which looks at all the vertices at once.\\

\noindent
\textbf{Expected:} since Field D* can deal with more headings it should produce a shorter path and therefore pass through less vertices than D* Lite.\\

\noindent
\textbf{Output:} absolute number of vertex accesses wrote to the debug file.

\subsubsection{Evaluation}

\newpage

\subsection{Test 3: Execution Time}

\noindent
\textbf{Test 2:} is concerned with how the time it took to execute each planning step, this value will be the total time taken for all planning steps.\\

\noindent
\textbf{Expected:} Field D*'s authors claim that on average it takes 1.7 times as long to plan than D* Lite something close to this figure is expected but it is subject to the hardware.\\

\noindent
\textbf{Output:} absolute total planning time wrote to the debug file.

\subsubsection{Evaluation}

\subsection{Test 4: Heading Changes}

\subsubsection{Evaluation}

\section{Discussion}
Discuss the findings, highlight any particularly interesting trends or patterns, also mention any difficulties encountered during testing. State the significance of the results