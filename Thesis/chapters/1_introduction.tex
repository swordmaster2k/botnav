\chapter{Introduction}

%-------------------------------------------------------------------------------------------------------

\section{Background}

\noindent
Provide a context to the reader, talk about the project, where the idea came from (extension of 3rd Project), some examples of courier robots today. Summary of the big industry players like Google and Amazon. Why is the project unique. Most of this can be taken from the background provided with the project proposal and literature review submissions for Research Skills.\\

\noindent
The ability to select a path to a destination, negotiate obstacles, and anticipate the movement of other people is something that you probably take for granted. To a human these skills are a part of everyday life and do not at first appear to be particularly special, until you try to replicate them. Programming a machine to deal with rush hour traffic or road closures is surprisingly difficult but that has not stop us from trying \cite{MIT}.\\

%-------------------------------------------------------------------------------------------------------

\section{Motivation}

\noindent
This section is a combination of the \textit{Justifications and Benefits} and \textit{Feasibility} sections from the project proposal and should state the following:

\begin{itemize}
\item Primary reason for undertaking this project, probably academic.
\item Any potential benefits to society and ITB developing this technology has.
\item The fact it evolved from a 3rd project and reuses existing equipment.
\item Back all this up with sound evidence that suggests that this project has an achievable goal.
\end{itemize} 

%-------------------------------------------------------------------------------------------------------

\section{Scope Definition}

\noindent
What is the actual scope of this project? Defining what to exclude is just as important as what is included so state both of these here in bullet points:\\

\noindent
The scope of this project \textbf{includes}:
\begin{itemize}
\item Path Planning: calculating the shortest and safest path to the delivery point.

\item State everything else...
\end{itemize}

\noindent
The following are \textbf{outside} the scope of this project:
\begin{itemize}
\item Collection: purely automated collection using beacons or computer vision. Goods will be manually handled by a human operator.

\item State everything else...
\end{itemize}

%-------------------------------------------------------------------------------------------------------

\section{Research Question(s)}

\noindent
State the primary research question of this project on its own line possibly in bold or italics. Then follow it with other sub research questions in bullet points:

\begin{itemize}
\item Where can mobile delivery robots be used outside of research laboratories?

\item State everything else...
\end{itemize}

%-------------------------------------------------------------------------------------------------------

\newpage

\section{Objectives}

\noindent
One line introducing the purpose of these objectives followed by the objectives themselves again in bullet points:

\begin{itemize}
\item To evaluate the effectiveness of path planning techniques in known and partially known environments.

\item State everything else...
\end{itemize}

%-------------------------------------------------------------------------------------------------------


